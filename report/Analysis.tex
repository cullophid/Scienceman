%!TEX root = /Users/andreasmller/Documents/DTU/Computerspil/scienceman/report/report.tex
\section{Analysis} % (fold)

The proof-of-concept prototype allows us to get a feel of how the different powers work, which is the most essential part of the gameplay. It has allowed us to tweak different aspects of the physics objects, in order to give the combat a more fluently feel. 
\subsection{Playability} % (fold)
	After playing the game for a while you realize just how endless the players possibilities are. Since all combat is based on simulation, instead of hardcoded attacks, no two combat scenarios will ever be the same. In fact playing the same encounter over and over will be a new experience every time. 



\subsection{Feedback} % (fold)
We persuaded a few friends to test our game prototype. Our intro level at the time had an enemy placed right next to the players starting point, which caused some problems, since it did not allow the player to become familiar with the game mechanics.  Once the testers were given the time to try out the different powers the tester quickly picked up the game, and went on to crush all the badguys we threw at them. We observed as they found creative ways to kill their enemies, that we had not event thought possible.

Mads Ingwar, one of the testers had this to say:

\quote{"The game starts in-medias-res, which sets you straight in the action. This along with a very aggressive enemy AI makes for quite an action surge straight off the bat.

The game's selling point is the unlimited fun to be had with physic powers. The interaction with the crates work seamlessly, and is very intuitive. It is especially nice that small tricks such as pulling a crate up and then forcepushing it towards the enemy at kill-speed works as expected. Furthermore the gravity well implementation is really nice both visually (some abstraction may apply) and in game terms. 

Overall the game is very addictive and I was soon lacking both enemies and more levels".}


\subsection{Reflection} % (fold)
One thing we constantly observed, was the unpredictability of both the players, and the game it self. The simulation based gameplay combined with the creativity of the players result in an infinite number of solutions to any game situation. An important strategy for the future development of the game will therefore be to avoid limiting the player, and make sure that every problem in the the game can have several solutions. Every level in the game must therefore also serve as a playground for exploring the physics powers.